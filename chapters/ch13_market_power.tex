\chapter{Market Power}
So far, we have focused on markets with \emph{perfect competition}. Recall that in section \ref{sec:single_firm_setup}, we defined markets as being perfectly competitive if no every agent is a \vocab{price-taker}. That is, every firm assumes that the price is fixed and that their choices do not affect the market price. This is typically the case where there are many buyers and sellers, so each agent's choices have negligible effect on the market price. 

Note however, that being in perfect competition does \emph{not} mean that the agents decisions do not have some effect on the market price, as in the case of general equilibrium analysis from \ref{ch:general_equilibrium}. In general equilibrium with a finite number of agents, every agent's choices have an effect on the market price. However, the key is that each agent \emph{behaves} as if the market price is fixed. That is, they do not take into the account the fact that their choices can change the price. This precludes, for example, producing less to restrict demand in order to raise the price per unit of a good sold. 

In this chapter, we drop the assumption that all agents are price takers, and instead analyze the case where firms can affect and, importantly, \emph{behave} as if they can affect the market price based on their production decisions. In cases where agents make choices taking into account that their choices will affect the market price, we say that such agents have \vocab{market power}.
