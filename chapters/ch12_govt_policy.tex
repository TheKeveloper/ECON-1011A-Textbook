\chapter{Government Policy and Externalities}
We have primarily examined what might what be called ``free market economies,'' where agents act without any government policy. In particular, we have acted under the assumption that one agent's consumption and behavior cannot affect another agent's behavior except through the market mechanisms. However, in the real world, certain behaviors impose externalities upon other agents. In order to examine the policy responses to these externalities, we need to understand how to model government policymaking, taking into account how agents will respond to the policy, and what types of objectives governments might have. 


\section{Modeling Government Policy} \label{sec:govt_policy}
Our first goal is to understand how to model and optimize government policies. The basic structure for optimizing government policies is as follows:
\begin{enumerate}
    \item Decide which policy variables the government will be able to manipulate (taxes, subsidies, etc).
    \item Agents in the economy optimize their behavior taking the government's policy variables as exogenous. 
    \item The government chooses the policy variables to optimize some objective function, taking the agent's responses as given. 
\end{enumerate}

\subsection*{Maximizing tax revenue}
We will make these steps concrete with a simple example of maximizing tax revenue via a tax on firms. Often in our political discourse, it is assumed that you are able to raise revenue simply by raising taxes. While this is often the case, because taxes distort behavior, the higher the taxes, the less firms will produce and hence it will decrease tax revenues. To see that this would be the case, suppose that governments can only tax revenue (not profits), and the government levies a 100\% tax on revenues. Because firms have other costs, they will choose to simply not produce, which means that the government's tax revenue would be 0. This is commonly known as the \vocab{Laffer curve} argument, which was popularized by economist Arthur Laffer, and was used during the Reagan administration (as well as subsequent administrations) to argue that cutting taxes would in fact raise revenue. 

Here, we will consider the problem of how to determine the optimal tax, by following the above steps.

\begin{enumerate}
    \item First, we will determine the government's policy variable. We will let $\tau$ be a tax on revenues, so that if the exogenous price of the good is $p$, and the firm produces $Q$ units of the good, the government receives,
    \begin{align*}
        (1 - \tau) p Q
    \end{align*}
    Notice that we could also interpret this as a percentage tax on each unit of the good sold. This type of tax is known as an \vocab{\emph{ad valorum}}, or proportional, tax. 
    \item Next, we solve the firm's profit maximization problem. The firm takes $\tau$ as given, has cost function $C(Q)$, and hence solves,
    \begin{align*}
        \max_{Q} (1 - \tau) p Q - C(Q)
    \end{align*}
    The familiar first order condition defines the optimal production,
    \begin{align*}
        (1 - \tau) p = C'(Q^*)
    \end{align*}
    \item Finally, we maximize the government's objective function, which in this case is tax revenue. We take the firm's choice of production, $Q^*$ So the government solves,
    \begin{align*}
        \max_{\tau} \tau p Q^*(\tau)
    \end{align*}
    This yields first order conditions for the optimal tax, $\tau^*$, 
    \begin{align*}
        \tau^* \frac{dQ^*}{d\tau} = -Q^*
    \end{align*}
    Note that this does not depend directly on price, as multiplying by price is just a monotonic transformation of the objective function. However, price may enter in implicitly through $Q^*$ and $\frac{dQ^*}{d\tau}$. Dividing through by $-Q^*$ yields
    \begin{align*}
        -\frac{\tau^*}{Q^*} \frac{dQ^*}{d\tau} = 1
    \end{align*}
    Notice that on the left, we have the elasticity of $Q$ with respect to $\tau$, which is the elasticity of production with respect to the tax. This tells us that at an optimum, the elasticity of supply with respect to the tax rate must equal 1. Intuitively, if we increase taxes by $1\%$ and our quantity (hence revenue) decreases by $1\%$, then we cannot earn any additional revenue by changing the tax because the decrease in revenue exactly balances out the increase in per unit tax. 
\end{enumerate}

\section{Modeling Externalities}
In all of the models that we have considered so far, we have assumed that each agent's utility depends only on the decisions that they make. We showed in the last chapter that in a setting where agents have full information and each agent's choices only directly affects their own utility, then a competitive equilibrium is efficient. In this section, we will examine how to model \vocab{externalities}, which is when the choices of one agent enter directly into the utility function of another agent. Perhaps the most common example of an externality is pollution. When you drive a car, the carbon dioxide emitted from the car not only affects your utility, but the utility of every individual. While externalities are normally thought of as negative, there are also positive externalities, like volunteering to clean up a local park making the park going experience for everyone else better. Of particular interest will be how the government can intervene in the case of externalities to make a Pareto improvement, that is, everyone will be better off than without government intervention. 

It is important to note however, that when we refer to externalities, we mean only when the choice of one agent enters \emph{directly} into the utility of another agent. We do not account for how an agent's choices can affect other agents through the market mechanism. For example, if you decide to buy oranges, increasing the demand for oranges and hence raising the price, and hence decreasing the utility for other orange buyers. While your choices in this case may have affected the utility of other agents, they did so only via market effects rather than direct effects, and hence we do not treat them the same as externalities with direct effects. 

\subsection*{Simple pollution model}
One of the canonical examples of an externality involves pollution. In the case of pollution, individuals gain utility from consuming some pollution generating good. For example, people can gain utility from driving a car, turning on the air conditioner, or charging your laptop, but all of this consumption involves producing carbon. The carbon then has a negative effect on the utility of other individuals in society through air pollution and climate change. 

The key is that the choices of each agent enters into the utility of every other agent. To see how this might be modeled, consider the following simple model of pollution:
\begin{itemize}
    \item There are $N$ identical agents.
    \item Agents have income $Y$.
    \item Agents choose between consuming a numeraire good $X$ and a carbon producing good $C$, where the utility from $C$ is strictly increasing and concave. The price of $X$ is 1, and the price of $C$ is exogenously given as $p$. 
    \item The carbon producing good generates negative externalities, where each agent's utility is negatively related to the average amount of carbon generated by the other agents. 
    \item The utility for agent $j$ is given by:
    \begin{equation*}
        X_j + U(C_j) - k \frac{\sum_{i \neq j} C_i}{N - 1}
    \end{equation*}
    where $k > 0$ is some constant denoting the marginal harm from pollution. 
    \item The government levies an \emph{ad valorum} (per unit) tax $\tau$, and the tax revenue is rebated to citizens in a lump sum, $L$, where $L = \frac{\sum_{i = 1}^N \tau C_i}{N}$. However, we assume that citizens do not internalize the effect of their taxes on $L$. \footnote{This means that agents treat $L$ as exogenous. They do not maximize their utility knowing that consuming more carbon will result in slightly higher tax rebates. This is realistic in the case where $N$ is very large, as any individual's contributions to the total tax revenue will have a negligible effect on their tax rebate.}
    \item Assume that the government's objective is to maximize average utility. Since all $N$ individuals are identical, this also will maximize the utility of every citizen. 
\end{itemize}
Our goal will be to choose the optimal tax rate, $\tau^*$ the maximizes the average citizen's utility. Recall from section \ref{sec:govt_policy} that to solve a government optimization problem, we first maximize the utility of each individual treating the government's choice variables as exogenous. 

\subsubsection*{The individual's maximization problem}
From the perspective of individual $j$, taxes, the lump sum rebate, and the consumption choices of other individuals $i$ is exogenous. Individual $j$'s maximization problem is therefore:
\begin{equation*}
    \max_{X_j, C_j} X_j + U(C_j) - k \frac{\sum_{i \neq j} C_i}{N - 1} \text{ s.t. } X_j + (p + \tau) C_j = Y + L
\end{equation*}
Substituting for the budget constraint\footnote{Substitute in $X_j = Y + L - p C_j$.} yields the unconstrained maximization problem: 
\begin{equation*}
    \max_{C_j} Y + L - (p + \tau) C_j + U(C_j) - k \frac{\sum_{i \neq j} C_i}{N - 1}
\end{equation*}
The only remaining choice variable for the the individual is now $C_j$. Taking first order conditions with respect to $C_j$ yields
\begin{align*}
    p + \tau = U'(C_j)
\end{align*}
Let $C^*$ denote the quantity that satisfies the above first order conditions. Note that in this case, $L = \tau C^*$ since every individual contributes the same amount in taxes and just receives an equal amount in tax rebates. Then each individual's utility in equilibrium is given by
\begin{equation*}
    \begin{split}
        Y + L - (p + \tau) C^* + U(C^*) - k \frac{\sum_{i \neq j} C^*}{N - 1} &= Y + \tau C^* - (p + \tau) C^* + U(C^*) - k \frac{(N - 1)C^*}{N - 1} \\
        &= Y + U(C^*) - (p + k) C^*
    \end{split}
\end{equation*}

\subsubsection*{Government optimization with individual choices as functions}
From the perspective of the government, $C^*$ is a function of the chosen tax rate, $\tau$, which we will denote as $C^* = C(\tau)$ for simplicity. The government chooses taxes to maximize the utility of the average citizen in equilibrium. So, the government's maximization problem is,
\begin{equation*}
    \max_{\tau} Y + U(C(\tau)) - (p + k) C(\tau)
\end{equation*}
Differentiating with respect to $\tau$ to obtain first order conditions yields,
\begin{equation*}
    \left[U'(C(\tau)) - (p + k)\right]C'(\tau) = 0
\end{equation*}
It can be shown that $C'(\tau) \neq 0$ for any choice of $\tau$. So, the first order condition is only satisfied when
\begin{equation*}
    U'(C(\tau)) = p + k
\end{equation*}
While it might seem difficult at first to obtain a closed form solution for $\tau$ that satisfies the above equation, recall that $C(\tau)$ is \emph{defined} such that,
\begin{equation*}
    U'(C(\tau)) = p + \tau
\end{equation*}
Then to satisfy the first order condition, we must have $p + k = p + \tau$. This implies that the optimal tax rate is given by $\tau^* = k$. 

Notice that in the above model, government is able to make a strict Pareto improvement when $k > 0$. That is, so long as pollution has a negative effect on the utilities of other agents, the government can intervene and make everyone better off. This is important because it demonstrates a key failing of the First Welfare Theorem, namely that a competitive equilibrium may not be Pareto efficient if externalities are present. 

\subsection*{Maximizing Firm Profits with Externalities}
To obtain some further intuition about externalities and government policy under externalities, we will examine a canonical model of pollution involving two firms. Firm $A$ will be a polluter, while firm $B$'s production is affected by $A$'s pollution. For simplicity, we will assume that the price of both firms' goods is $p$. 
\begin{itemize}
    \item Firm $A$ makes candy, producing pollution as a byproduct. The firm chooses the quantity of candy, which we will denote by $A$, to produce. Firm $A$'s cost function is given by
    \begin{equation*}
        \frac{1}{2} c_a A^2
    \end{equation*}
    Assume that firm $A$ does not care about the impact of their pollution. The firm maximizes profit, which is given by
    \begin{equation*}
        \pi_A = pA - \frac{1}{2} c_a A^2
    \end{equation*}
    Solving the first order conditions yields the optimal choice of production as, 
    \begin{equation*}
        A^* = \frac{p}{c_a}
    \end{equation*}

    \item Firm $B$ cleans laundry, and laundry is made more costly by pollution. $B$'s cost function is given by
    \begin{equation*}
        \frac{1}{2} c_b B^2 + x A \text{ for $x > 0$ }
    \end{equation*}
    Firm $B$'s profit is therefore:
    \begin{equation*}
        \pi_B = pB - \frac{1}{2}c_b B^2 - x A
    \end{equation*}
    Since $B$ cannot affect $A$'s pollution, pollution can be treated essentially as a fixed cost for $B$. The first order condition yields $B$'s optimal production as
    \begin{equation*}
        B^* = \frac{p}{c_b} 
    \end{equation*}
\end{itemize}
Because we are dealing with two firms in this case, we can make sense of adding their profits to gain a measure of the total ``social'' profit. The total combined profit when firms optimize individually is therefore given by:
\begin{equation*}
    \begin{split}
        \pi_A^* + \pi_B^* &= \frac{p^2}{c_a} - \frac{p^2}{2 c_a} + \frac{p^2}{c_b}  - \frac{p^2}{2 c_b} - x \frac{p}{c_a} \\ 
        &= \frac{p (p - 2x) }{2 c_a} + \frac{p^2}{2 c_b}
    \end{split}
\end{equation*}

\subsubsection*{Joint Profit Optimization}
We can next consider what would happen if firms $A$ and $B$ were to merge into one firm. The goal for the new owner would then be to choose production quantities $A$ and $B$ to maximize total profit.
\begin{equation*}
    \pi_{A + B} = pA - \frac{1}{2} c_a A^2 + pB - \frac{1}{2} c_b B^2 - xA
\end{equation*}
Solving for the optimal quantities yields,
\begin{equation*}
    A^{**} = \frac{p - x}{c_a} \text{ and } B^{**} = \frac{p}{c_b}
\end{equation*}
Notice that $A^{**} < A^*$. This is because the pollution externality is now being internalized by the joint firm. That is, the cost of pollution now affects the decision maker's profit rather than affecting another firm, which effectively increases the cost of producing $A$, hence reducing the quantity produced. Notice however that $B^{**} = B^*$, because in both cases, the cost of pollution is effectively a fixed cost from the perspective of choosing how much $B$ to produce, and hence the optimal quantities are the same. 

Under joint optimization, the total profit becomes
\begin{equation*}
    \pi_{A + B}^{**} = \frac{(p - x)^2}{2 c_a} + \frac{p^2}{2 c_b}
\end{equation*}
First, even before doing any actual calculations, we know that the total profits under joint optimization must be at least the profits under separate optimization. This is by definition because we have chosen $A^{**}$ and $B^{**}$ to optimize the total profit. To determine the exact difference, we can simply subtract the two profit terms from each other to obtain,
\begin{equation*}
    \pi_{A + B}^{**} - (\pi_A^* + \pi_B^*) = \frac{x^2}{2c_a}
\end{equation*}
This is positive for any $x \neq 0$, which is precisely the case where there exists an externality from one firm to another. 

If we treat the total firm profits as some measure of social welfare, this model suggests that total social welfare can be improved when the firms are combined than when they are separate in the case of externalities. If externalities between the two firms did not exist, then there should be no difference in total profit between jointly optimizing and separately optimizing firms. This suggests that if we could somehow enforce that the firms behave as if they were jointly optimizing, then we could improve total social welfare. We will discuss how government policymakers can try to achieve this optimal social welfare without forcing firms to jointly optimize in the next section. 
