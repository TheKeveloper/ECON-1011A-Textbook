\chapter{General Equilibrium}
While we've done a lot of economics so far, we have not actually modeled a full economy yet! We've instead focused on studying what is known as \vocab{partial equilibrium}, which examines how agents optimize their behavior taking the other side of the market. For example, when we study the firm optimization problem, we take the price of the good as given and assume there are some consumers who are willing to buy the goods at that price. Similarly, when we studied consumer theory, we assumed that consumers could always buy goods at a given price, and we do not model how the firms actually produce those goods. In both cases, we did not concern ourselves with how prices or wages are set. However, in real economies, prices depend on how agents in the economy behave and the decisions that they make. In this section, we will model a \vocab{general equilibrium}, which considers how individual optimizations leads supply and demand in markets clearing through appropriately set prices. In a general equilibrium model, rather than prices being exogenous, they will be endogenously determined by supply and demand, and the only truly exogenous variables will be the so-called economic ``primitives,'' which are coefficients that determine an individual's utility function, productivity, etc., and we will have truly modeled a full economy. 