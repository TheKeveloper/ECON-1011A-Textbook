\chapter{Expenditure Minimization}

Recall that in our discussion of firm theory, we noted that the firm's problem of profit maximization is the dual problem of cost minimization. That is, if we were to determine the set of inputs that maximized profits, we would get the same result if we knew this maximum profit level and then determined the set of inputs that minimized the costs in order to achieve that level of profit. 

Analogously, in the case of consumer theory, the dual problem of utility maximization is known as \vocab{expenditure minimization.} While utility maximization asks for the set of goods needed to maximize utility under some constraint (e.g. what is the most utility I can achieve with \$20?), expenditure minimization asks for the set of goods that most cheaply attains a given level of utility (e.g. what is the cheapest way to achieve 100 utils of utility?). This chapter will study the expenditure minimization problem and explore how examining the utility maximization and expenditure minimization problems together gives us a fuller picture of how consumer behavior reacts to prices.

\section{Problem Setup}

Formally, suppose we have goods $1, \ldots, n$ with prices $p_1, \ldots, p_n$, respectively, and our choice variables $x_1, \ldots, x_n$ correspond to how much we purchase of each good, respectively. Denote our utility function from these goods as $u(x_1, \ldots, x_n)$. Suppose we must attain a utility level of at least $\bar{u}$, where $\bar{u}$ is some exogenous variable that is given to us. What is the cheapest way to achieve $\bar{u}$? We can write down our expenditure minimization problem as 
$$\min _{x_{1}, x_{2}, \ldots, x_{n}} \sum_{i=1}^{n} p_{i} x_{i} \text { s.t. } u\left(x_{1}, x_{2}, \ldots, x_{n}\right) \geq \bar{u}.$$
If we use vector notation and denote $\vec{p}$ as our prices and $\vec{x}$ as our chosen quantities, we can write this more succinctly as
$$\min _{\vec{x} \geq 0} \vec{p} \cdot \vec{x} \text { s.t. } u(\vec{x}) \geq \bar{u}$$

We can solve this problem similar to how we solved the cost minimization problem from our study of firm theory.

\subsection*{First Order Conditions}
\TODO{Standardize whether we add or subtract the Lagrangian term}

We can write our Lagrangian in vector form as
$$\mathcal{L}(\mathbf{x}, \lambda)=\mathbf{p} \cdot \mathbf{x}+\lambda[\bar{u}-u(\mathbf{x})].$$
Our first order conditions can be written as 
$$
\begin{cases}
\mathcal{L}_{x_{i}}=p_{i}-\lambda^{*} u_{x_{i}}\left(\mathbf{x}^{*}\right)=0 \text { for all } i \\
\mathcal{L}_{\lambda}=\bar{u}-u\left(\mathbf{x}^{*}\right)=0.
\end{cases}.$$
Notice how this looks similar to our conditions from utility maximization. The last condition is simply the requirement that we attain $\bar{u}$. Our first set of conditions regarding each $\mathcal{L}_{x_{i}}$ term is the same as what we had from utility maximization, except since our constraint is different, the values of $\lambda^*$ and $\vec{x}^*$ are different that what they were in the utility maximization case. Note that we could also have written this first set of conditions in vector form as
$$\mathbf{p}=\lambda^{*} \nabla u\left(\mathbf{x}^{*}\right)$$

Recognize that our first order conditions give us $n+1$ linear equations with $n+1$ unknowns. We thus have a solution for $\lambda^*$ as well as solutions for
$$x_i^*(\vec{p}, \bar{u})$$
for $i = 1, 2, \ldots, n$. Notice that in the case of utility maximization, our solutions were of the form $x_i^*(\vec{p}, y)$. The important difference now is that our solution to the expenditure minimization problem is in terms of the required utility level rather than budget. Whereas we commonly denote $x_i^*(\vec{p}, y)$ as Marshallian demand, the solution to the utility maximization problem, we will denote the solution we just derived for the expenditure minimization problem as 
$$h_i(\vec{p}, \bar{u})$$
for $i=1, 2, \ldots, n$. This expression is known as the \vocab{Hicksian demand}, also referred to as the \vocab{compensated demand}. Moving forward, we will often drop arguments and simply write $x_i^*$ and $h_i^*$ for Marshallian and Hicksian demand, respectively, but it is important to remember how the arguments of these functions differ.

\section{Expenditure Function}

We now turn to the value function for the expenditure minimization problem, which is called the \vocab{expenditure function}:
$$e(\mathbf{p}, \bar{u})=\mathbf{p} \cdot \mathbf{h}(\mathbf{p}, \bar{u})=\sum_{i=1}^{n} p_{i} h_{i}(\mathbf{p}, \bar{u}).$$

In the last chapter, we noted that there are some similarities between a consumer's indirect utility function from utility maximization and a firm's profit function from profit maximization. However, the analogy was not exact. First, while the profit function is convex in prices, the indirect utility function is quasi-convex in prices. Second, while the profit function is homogeneous of degree 1, the indirect utility function is not.

With the expenditure function, we will notice that it is an exact analogy to a firm's cost function from cost minimization. We encourage the reader to refer back to Section~\ref{sec:cost_properties} to notice the parallels between properties of the cost function and the expenditure function.

\subsection*{Properties of the Expenditure Function}

\begin{description}
\item[Shephard's Lemma] $\pdv{e}{p_i}(\vec{p}, \bar{u}) = h_{i}(\mathbf{p}, \bar{u})$. In the case of cost minimization, we saw that the derivative of cost with respect to input prices was equal to the input demand. Analogously, in the case of expenditure minimization, the derivative of expenditure with respect to prices of goods is equal to the Hicksian demand of those goods. The lemma and the proof are the exact same as what we saw before.

\begin{proof}
Applying the constrained envelope theorem, we have
$$\pdv{e}{p_i}(\vec{p}, \bar{u}) = \frac{\partial}{\partial p_i}\left\{\mathbf{p} \cdot \mathbf{h}+\lambda^{*}[\bar{u}-u(\mathbf{h})]\right\} = h_{i}(\mathbf{p}, \bar{u}).$$
\end{proof}

\item[Homogeneous of degree 1 in prices] $e(\alpha\vec{p}, \bar{u}) = \alpha e(\vec{p}, \bar{u})$. The proof is the same as in Section~\ref{sec:cost_properties} and is encouraged as an exercise for the reader. The intuition is also the same: if we changed the units of the prices of all goods, then the units of our expenditure function would scale accordingly.

\item[Concave in prices] For all $\alpha \in [0, 1]$ and any price vectors $\vec{p}_1$ and $\vec{p}_2$, 
$$e(\alpha \vec{p}_1 + (1-\alpha)\vec{p}_2, \bar{u}) \geq \alpha e(\vec{p}_1, \bar{u}) + (1-\alpha) e(\vec{p}_2, \bar{u}).$$
Again, the proof is the same as in Section~\ref{sec:cost_properties} and is encouraged as an exercise for the reader. The basic idea of the proof is that at both $(\vec{p}_1, \bar{u})$ and $(\vec{p}_2, \bar{u})$, purchasing $\vec{h}(\alpha \vec{p}_1 + (1-\alpha)\vec{p}_2, \bar{u})$ will achieve the utility constraint, but you can spend weakly less by purchasing $\vec{h}(\vec{p}_1, \bar{u})$ or $\vec{h}(\vec{p}_2, \bar{u})$ at these points, respectively. A similar interpretation follows as well: as a consumer, you would rather have prices fluctuate than have them stay constant at their average value, since you can reoptimize your consumption bundle at each of the fluctuations to achieve a lower expenditure.

\item[Non-decreasing in prices and utility] If $\vec{p}$ of $\bar{u}$ increase, then $e(\vec{p}, \bar{u})$ weakly increases. The proof is straightforward: if prices or required utility fell from a previous value, then the previous bundle would still satisfy the constraint, so minimum expenditure cannot rise.

\item[Continuous] We will not prove this property of the expenditure function, but it is good to remember.

\end{description}

\TODO{Add example of Cobb Douglas}

\section{Measuring Price Changes}

One important application of our study of the expenditure function is a metric for measuring price changes, or a \vocab{price index}. Prices for goods often change, but what would be a good way to aggregate different price changes across multiple goods into a single metric?

One appealing approach is to use the expenditure function. Suppose we start with utility $u_0$ when prices are at $\vec{p}^0$, and then prices change to $\vec{p}^1$. A natural metric might measure how much expenditure would need to change in order to maintain utility $u_0$ at this new price level $\vec{p}^1$. Mathematically, this definition would look like
$$\frac{e(\vec{p}^1, u_0)}{e(\vec{p}^0, u_0)}.$$
As another way to interpret this definition, if we had income $y_0$ when prices were $\vec{p}^0$, and then prices changed to $\vec{p}^1$ and our income changed to $\frac{e(\vec{p}^1, u_0)}{e(\vec{p}^0, u_0)}y_0$, then our utility would be the constant.

The definition above is wonderful in theory but cannot be measured in practice: since we cannot observe consumers' utilities, we also cannot observe their expenditure functions. However, one workaround is to recognize that 
$$e(\vec{p}, u) = \vec{p}\cdot \vec{h} (\vec{p}, u)$$
by definition, and that
$$\mathbf{x}(\mathbf{p}, y)=\mathbf{h}(\mathbf{p}, v(\mathbf{p}, y))$$
by duality. It is important to understand why the equation above holds: the utility-maximizing consumption given a budget (Marshallian demand) is the same as the expenditure-minimizing consumption to achieve this realized utility level (Hicksian demand). This observation is useful because although we cannot measure the expenditure function, we can measure Marshallian demand, since $\vec{x}, \vec{p}$, and $y$ are all observed.

Using these formulations, we can approximate 
$$\frac{e\left(\mathbf{p}^{1}, u_{0}\right)}{e\left(\mathbf{p}^{0}, u_{0}\right)} \approx \frac{\sum_{i=1}^{n} p_{i}^{1} x_{i}^{0}}{\sum_{i=1}^{n} p_{i}^{0} x_{i}^{0}},$$
where the approximation on the right-hand side is known as the \vocab{Laspeyres price index}. Notice where this approximation is not exact. The denominators are equal, since 
$$e\left(\mathbf{p}^{0}, u_{0}\right) = \sum_{i=1}^{n} p_{i}^{0} x_{i}^{0}$$
by definition of the expenditure function, but the numerators differ:
$$e\left(\mathbf{p}^{1}, u_{0}\right) \leq \sum_{i=1}^{n} p_{i}^{1} x_{i}^{0}.$$
Intuitively, this is because we know $\vec{x}^0$ achieves our minimum utility $u_0$, but purchasing this same bundle might no longer be the minimum expenditure to achieve this utility once prices change. Thus, 
$$\frac{e\left(\mathbf{p}^{1}, u_{0}\right)}{e\left(\mathbf{p}^{0}, u_{0}\right)} \leq \frac{\sum_{i=1}^{n} p_{i}^{1} x_{i}^{0}}{\sum_{i=1}^{n} p_{i}^{0} x_{i}^{0}},$$
so the Laspeyres price index is an overestimate of our theoretical ideal. If we had an income $y_0$ initially and multiplied it by the Laspeyres price index after the price change, we would be able to weakly increase our utility, since we can still afford our old consumption bundle, but we may be able to reoptimize and do even better.

A similar approach to the same problem is the \vocab{Paasche price index}
$$\frac{e\left(\mathbf{p}^{1}, u_{1}\right)}{e\left(\mathbf{p}^{0}, u_{1}\right)} \approx \frac{\sum_{i=1}^{n} p_{i}^{1} x_{i}^{1}}{\sum_{i=1}^{n} p_{i}^{0} x_{i}^{1}},$$
where we now weight prices by $\vec{x}^1$ instead of $\vec{x}^0$. The theoretical ideal on the left is a similar idea but uses the new utility level as the baseline rather than the old utility level. The Paasche price index on the right is now an underapproximation of the theoretical ideal on the left, since while the numerators are equal, the denominators differ as
$$e\left(\mathbf{p}^{0}, u_{1}\right) \leq \sum_{i=1}^{n} p_{i}^{0} x_{i}^{1}$$
by a similar reasoning to before. Intuitively, the Laspeyres price index measures how much a consumption bundle that was optimal in the base year costs now, whereas the Paasche price index measures how much a consumption bundle that is optimal now would have cost in the base year. 

\section{Consequences of Duality}
Before starting the derivation, we highlight two important consequences of the duality of the utility-maximization and the expenditure-minimization problems. 
\begin{description}
\item[Given prices, the indirect utility function and the expenditure function are inverses.]
We observe that
$$v(\vec{p}, e(\vec{p}, \bar{u})) = \bar{u}.$$
Intuitively, we defined $e(\vec{p}, \bar{u})$ as the minimum cost to achieve $\bar{u}$, so if we have a starting budget of $e(\vec{p}, \bar{u})$, then the maximum utility we can achieve with this budget is $\bar{u}$.
Similarly, we observe that 
$$e(\vec{p}, v(\vec{p}, y)) = y.$$
This time, we defined $v(\vec{p}, y)$ to be the maximum utility we can achieve with budget $y$, so the minimum budget needed achieve utility level $v(\vec{p}, y)$ must be $y$. We conclude that if prices $\vec{p}$ are given, then the value functions $e$ and $v$ are inverses of each other.

\item[At optimality, Marshallian and Hicksian demand coincide.] 
We observe that 
$$\mathbf{x}(\mathbf{p}, e(\mathbf{p}, \bar{u}))=\mathbf{h}(\mathbf{p}, \bar{u}).$$
Intuitively, $\vec{h}(\vec{p}, \bar{u})$ is the cheapest bundle to attain utility $\bar{u}$, and it costs $e(\vec{p}, \bar{u})$. Thus, if we had a budget of $e(\vec{p}, \bar{u})$, then most satisfying bundle to buy should also be $\vec{h}(\vec{p}, \bar{u})$.
Similarly, we observe that
$$\mathbf{h}(\mathbf{p}, v(\mathbf{p}, y))=\mathbf{x}(\mathbf{p}, y).$$
Now, $\vec{x}(\vec{p}, y)$ is the most satisfying bundle to buy with a budget $y$, and it gives us utility $v(\mathbf{p}, y)$. Thus, if we must achieve a utility of $v(\mathbf{p}, y)$, then the cheapest bundle that attains this utility should also be $\vec{x}(\vec{p}, y)$. 

To summarize, since utility-maximization and expenditure-minimization both yield optimal consumption bundles, these bundles coincide when our budget from the utility-maximization problem is just enough to attain the required utility from the expenditure-minimization problem. 
\TODO{I left out the geometric interpretation because I think it's more for the interested reader than it is helpful for understanding, but I might add this back in?}
\end{description}

\section{Measuring welfare changes}
One of the challenges in dealing with utility in economics is that it is not directly observable. However, it would be nice to be able to quantify the change in utility from a change in prices. Fortunately, the expenditure function offers us a nice way to do so.

\subsection*{Money Metric Utility}
The money metric utility function offers us a way to measure utility changes in dollars, which is observable in the real world. In particular, it allows us a way to map changes in prices to a change in income. Suppose we have some utility function $u$ that we want to be able to measure in dollars. Take two utility levels, $\bar{u}_1, \bar{u}_2$ such that $\bar{u}_1 > \bar{u}_2$. Recall that for any set of prices, $\vec{r}$, we have
\begin{align*}
    e(\vec{r}, \bar{u}_1) > e(\vec{r}, \bar{u}_2)
\end{align*}
This tells us that $e(\vec{r}, \cdot)$ is a monotonic transformation in its second argument, so it can be applied to any valid utility function to create a new utility function. So, for any allocation $\vec{x}$, we can define the \vocab{money metric utility function} with reference prices $\vec{r}$ as,
\begin{align*}
    m(\vec{x}) = e(\vec{r}, u(\vec{x}))
\end{align*}
The idea here is that we can measure how much utility we have for a given allocation by examining what is the minimum amount that we would need to spend to achieve the utilty for that allocation. 

This gives us a natural way to compare utilities for two different sets of prices. Consider prices $\vec{p}$ and $\vec{q}$. For a fixed level of income $y$, we can compare their utilities through their respective value functions, $v(\vec{p}, y)$ and $v(\vec{q}, y)$. We can then convert these utility values into a money metric values to measure how much better (or worse) in dollars $\vec{p}$ is compared to $\vec{q}$
\begin{align*}
    m(v(\vec{p}, y)) - m(v(\vec{q}, y)) = e(\vec{r}, v(\vec{p}, y) - e(\vec{r}, v(\vec{q}, y))
\end{align*}
If the above expression is positive, then $\vec{p}$ is preferred to $\vec{q}$, and if it is negative, then $\vec{q}$ is preferred to $\vec{p}$. Importantly, these quantities are ones that we could in principle measure, by asking or observing how much money an individual requires at prices $\vec{r}$ to be indifferent with having prices $\vec{p}$ and income $y$. 

\subsection*{Equivalent and compensating variation}
So far, we have used the money metric utility function to measure the differences between two sets of prices, $\vec{p}$ and $\vec{q}$, with respect to some set of reference prices $\vec{r}$. However, there is a question of how we select this reference price $\vec{r}$. Two reasonable prices we might pick are either $\vec{p}$ or $\vec{q}$. For the sake of this analysis, we will refer to $\vec{p}$ as the \emph{old} prices and $\vec{q}$ as the \emph{new} prices.

\subsubsection*{Equivalent Variation}
The \vocab{equivalent variation (EV)} tells us the welfare effect of a price change from $\vec{p}$ to $\vec{q}$ through an equivalent change in income at \emph{old} prices, $\vec{p}$. Mathematically, we are setting the reference price $\vec{r} = \vec{p}$,
\begin{align*}
    EV = e(\vec{p}, v(\vec{q}, y)) - e(\vec{p}, v(\vec{p}, y)) = e(\vec{p}, v(\vec{q}, y)) - y
\end{align*}
Intuitively, we can think of EV as how much you would be willing to pay to keep prices at $\vec{p}$ instead of $\vec{q}$. An easy way to remember this is that the equivalent variation tells you what would have been the \emph{equivalent} change in income for a given price change.

\subsubsection*{Compensating Variation}
The \vocab{compensating variation (CV)} tells us the welfare effect of a price change from $\vec{p}$ to $\vec{q}$ through an equivalent change in income at \emph{new} prices, $\vec{q}$. Mathematically, we are setting the reference price $\vec{r} = \vec{q}$,
\begin{align*}
    CV = e(\vec{q}, v(\vec{q}, y)) - e(\vec{q}, v(\vec{p}, y)) = y - e(\vec{q}, v(\vec{p}, y))
\end{align*}
Intuitively, we can think of CV as how much you would be willing to pay to change prices from $\vec{p}$ to $\vec{q}$. Or equivalently, it is the negative of how much you would need to be paid after a price change to make you just as happy as you were before. This second interpretation is useful for remembering the definition of compensating variation, as it tells you how much you need to be \emph{compensated} to be just as happy for a given price change. 

\subsubsection*{Hicksian Integral}
The equivalent and compensating variations have an interesting connection to the Hicksian demand function when we consider the change in only one price.  Suppose that we have a single price change, from $p$ to $q$.

First, we will examine the equivalent variation, which is
\begin{align*}
    EV = e(p, v(q, y)) - e(p, v(p, y)) = e(p, v(q, y)) - y
\end{align*}
However, we also have that $y = e(q, v(q, y))$. So we can rewrite the equivalent variation as,
\begin{align*}
    EV = e(p, v(q, y)) - e(q, v(q, y))
\end{align*}
Next, notice that for an arbitrary price $s$, we have by Shephard's Lemma\TODO{Add reference},
\begin{align*}
    \frac{\partial e(s, v(q, y))}{\partial s} = h(s, v(q, y))
\end{align*}
Where $h$ is the Hicksian demand for the single good. Then, by the fundamental theorem of calculus, we have
\begin{align*}
    EV &= e(p, v(q, y)) - e(q, v(q, y)) \\
    &= \int_q^p \frac{\partial}{\partial s} e(s, v(q, y)) \, ds \\
    &= \int_q^p h(s, v(q, y)) \, ds
\end{align*}

We can perform a similar process with the compensating variation to obtain,
\begin{align*}
    CV &= e(p, v(p, y)) - e(q, v(p, y)) \\
    &= \int_q^p \frac{\partial}{\partial s} e(s, v(p, y)) \, ds \\
    &= \int_q^p h(s, v(p, y)) \, ds
\end{align*}

Note that while we handled these in the single good case, it generalizes to multiple goods. A simple way to see this is to consider multiple goods, but only changing the price of one good. Then we obtain the same results as above except the integral is with respect to the price of that specific good, and the the Hicksian is the Hicksian demand of that specific good. 

\section*{Recap}