\chapter{Firms with a single input}
We start with one of the simplest economic models: a firm in a perfectly competitive market with a single input. Firms are one of the most basic components of the economy. Firms purchase inputs, convert those inputs into outputs via a production function, and then sell those goods to make a profit. This also gives firms a clear objective function: profit. To make things simple, we assume that the firm only requires one input to produce their good, so the only choice that the firm makes is how much input to purchase. 

\section{Model setup}
We will formalize a mathematical model of the firm. We assume that the firm can choose to hire $\ell$ units of labor at a wage $w$, which is the price per unit of labor. The firm has a production function, $f(\ell)$, which takes the units of labor as an input, and returns some unit of product. We assume that the production function is continuous and twice differentiable, increasing, $f'(\ell) > 0$, and concave $f''(\ell) < 0$, for any of $\ell$. The firm can then sell each unit of product at a price $p$. The firm takes both $p$ and $w$ as exogenous variables. We can then define the firm's profit, $\pi$, as follows:
\begin{align}
    \pi(\ell; p, w) = p f(\ell) - w \ell 
\end{align}
Notice that our profit function has inputs. $\ell$, the units of labor hired, is the choice variable for the firm. $p$, the price of the product, and $w$, the wage cost of labor, are both exogenous variables. While we write them explicitly here, we will often only write $\pi$ and the arguments to it are implicit. 

While this seems like a fairly simple model, there are some pretty important assumptions underlying it.

\subsection{Assumptions}
\begin{description}
    \item[Perfectly competitive market for output] Notice that the firm treats the price $p$ as exogenous. That is, no matter how much the firm produces, they can always sell goods at price $p$, and \emph{only} at price $p$. This means, first, that the firm is a \vocab{price-taker}, which means that they cannot set a price $p$ that differs from the market price $p$. The underlying assumption here is that there are enough other firms that if this firm were to raise its price, all of the customers would buy from other firms and our firm would sell 0. This assumption also entails that the amount our firm produces does not affect the market price, which can be taken to mean that there are many other firms producing a lot of the same good, so our firm's decisions do not have a noticeable effect on $p$. 
    \item[Perfectly competitive labor market] Similar to the above market for goods, we also assume that the market for labor is perfectly competitive. That is, the firm can only hire at the wage $w$, and that no matter how much labor the firm hires, the wage will not change. 
    \item[No liquidity constraints] We assume that the firm has the ability to hire as much labor as they want, and all that matters is the final profit. That is, the firm does not have some fixed budget for labor at the beginning. This can be thought of as a firm's ability to borrow at zero interest to finance labor so long as the loan is paid back. This assumption is key to the firm problem, as it allows us to deal with an unconstrained maximization problem rather than requiring a budget constraint for the firm. 
    \item[Diminishing marginal returns to consumption] This was expressed mathematically as $f''(\ell) < 0$. In real terms, this says that each additional unit of labor contributes less additional production than the previous unit of labor did, and represents a sort of ``too many cooks in the kitchen'' effect. Notice however that we will assume $f'(\ell) > 0$, so even if each additional unit of labor contributes less additional output than the previous unit, adding more units of labor can never make us produce less output. 
\end{description}

One reasonable question to ask with all of these assumptions in place is whether they are realistic assumptions. The answer is that they probably are not all perfectly realistic. However, there are cases where these assumptions might be close enough. Consider the market for corn, for example. Each individual farmer's corn production has a negligible effect on the market as a whole, and they have enough money every year to grow as much corn as is profitable. However, we will see that even if these assumptions are not all realistic, they help simplify the model so that we can solve it and gain some useful insights about the mechanics of this economy. 

\section{Solving the model}